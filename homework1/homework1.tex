%%%%%%%%%%%%%%%%%%%%%%%%%%%%%%%%%%%%%%%%%
% Structured General Purpose Assignment
% LaTeX Template
%
% This template has been downloaded from:
% http://www.latextemplates.com
%
% Original author:
% Ted Pavlic (http://www.tedpavlic.com)
%
% Note:
% The \lipsum[#] commands throughout this template generate dummy text
% to fill the template out. These commands should all be removed when 
% writing assignment content.
%
%%%%%%%%%%%%%%%%%%%%%%%%%%%%%%%%%%%%%%%%%

%----------------------------------------------------------------------------------------
%	PACKAGES AND OTHER DOCUMENT CONFIGURATIONS
%----------------------------------------------------------------------------------------

\documentclass{article}

\usepackage{fancyhdr} % Required for custom headers
\usepackage{lastpage} % Required to determine the last page for the footer
\usepackage{extramarks} % Required for headers and footers

% Margins
\topmargin=-0.45in
\evensidemargin=0in
\oddsidemargin=0in
\textwidth=6.5in
\textheight=9.0in
\headsep=0.25in 

\linespread{1.1} % Line spacing

% Set up the header and footer
\pagestyle{fancy}
\lhead{\hmwkAuthorName} % Top left header
%\chead{\hmwkClass\ (\hmwkClassInstructor\ \hmwkClassTime): \hmwkTitle} % Top center header
\chead{\hmwkClass\ (\hmwkClassInstructor): \hmwkTitle} % Top center header
\rhead{\firstxmark} % Top right header
\lfoot{\lastxmark} % Bottom left footer
\cfoot{} % Bottom center footer
\rfoot{Page\ \thepage\ of\ \pageref{LastPage}} % Bottom right footer
\renewcommand\headrulewidth{0.4pt} % Size of the header rule
\renewcommand\footrulewidth{0.4pt} % Size of the footer rule

\setlength\parindent{0pt} % Removes all indentation from paragraphs

%----------------------------------------------------------------------------------------
%	DOCUMENT STRUCTURE COMMANDS
%	Skip this unless you know what you're doing
%----------------------------------------------------------------------------------------

% Header and footer for when a page split occurs within a problem environment
\newcommand{\enterProblemHeader}[1]{
\nobreak\extramarks{#1}{#1 continued on next page\ldots}\nobreak
\nobreak\extramarks{#1 (continued)}{#1 continued on next page\ldots}\nobreak
}

% Header and footer for when a page split occurs between problem environments
\newcommand{\exitProblemHeader}[1]{
\nobreak\extramarks{#1 (continued)}{#1 continued on next page\ldots}\nobreak
\nobreak\extramarks{#1}{}\nobreak
}

\setcounter{secnumdepth}{0} % Removes default section numbers
\newcounter{homeworkProblemCounter} % Creates a counter to keep track of the number of problems

\newcommand{\homeworkProblemName}{}
\newenvironment{homeworkProblem}[1][Problem \arabic{homeworkProblemCounter}]{ % Makes a new environment called homeworkProblem which takes 1 argument (custom name) but the default is "Problem #"
\stepcounter{homeworkProblemCounter} % Increase counter for number of problems
\renewcommand{\homeworkProblemName}{#1} % Assign \homeworkProblemName the name of the problem
\section{\homeworkProblemName} % Make a section in the document with the custom problem count
\enterProblemHeader{\homeworkProblemName} % Header and footer within the environment
}{
\exitProblemHeader{\homeworkProblemName} % Header and footer after the environment
}

\newcommand{\problemAnswer}[1]{ % Defines the problem answer command with the content as the only argument
\noindent\framebox[\columnwidth][c]{\begin{minipage}{0.98\columnwidth}#1\end{minipage}} % Makes the box around the problem answer and puts the content inside
}

\newcommand{\homeworkSectionName}{}
\newenvironment{homeworkSection}[1]{ % New environment for sections within homework problems, takes 1 argument - the name of the section
\renewcommand{\homeworkSectionName}{#1} % Assign \homeworkSectionName to the name of the section from the environment argument
\subsection{\homeworkSectionName} % Make a subsection with the custom name of the subsection
\enterProblemHeader{\homeworkProblemName\ [\homeworkSectionName]} % Header and footer within the environment
}{
\enterProblemHeader{\homeworkProblemName} % Header and footer after the environment
}
   
%----------------------------------------------------------------------------------------
%	NAME AND CLASS SECTION
%----------------------------------------------------------------------------------------

\newcommand{\hmwkTitle}{Homework\ \#1} % Assignment title
\newcommand{\hmwkDueDate}{Sunday,\ August\ 28,\ 2016} % Due date
\newcommand{\hmwkClass}{CSC\ 591} % Course/class
%\newcommand{\hmwkClassTime}{10:30am} % Class/lecture time
\newcommand{\hmwkClassInstructor}{Samatova} % Teacher/lecturer
\newcommand{\hmwkAuthorName}{Charles Haithcock (cehaith2)} % Your name

%----------------------------------------------------------------------------------------
%	TITLE PAGE
%----------------------------------------------------------------------------------------

\title{
\vspace{2in}
\textmd{\textbf{\hmwkClass:\ \hmwkTitle}}\\
\normalsize\vspace{0.1in}\small{Due\ on\ \hmwkDueDate}\\
%\vspace{0.1in}\large{\textit{\hmwkClassInstructor\ \hmwkClassTime}}
\vspace{0.1in}\large{\textit{\hmwkClassInstructor}}
\vspace{3in}
}

\author{\textbf{\hmwkAuthorName}}
\date{} % Insert date here if you want it to appear below your name

%----------------------------------------------------------------------------------------

\begin{document}

\maketitle

%----------------------------------------------------------------------------------------
%	TABLE OF CONTENTS
%----------------------------------------------------------------------------------------

%\setcounter{tocdepth}{1} % Uncomment this line if you don't want subsections listed in the ToC

\newpage
\tableofcontents
\newpage

%----------------------------------------------------------------------------------------
%	PROBLEM 1
%----------------------------------------------------------------------------------------

% To have just one problem per page, simply put a \clearpage after each problem

\begin{homeworkProblem}[Problem \#\arabic{homeworkProblemCounter}]
A customer asks you to provide the most efficient implementation of the following operations 
over a large graph (about 10,000 vertices): add/delete edges/vertices.

    \begin{homeworkSection}{1.}
        What questions would you ask the customer that will help you to make the right decision
        about the customer's request? [Hints: Check the types of graphs]\\
        \problemAnswer{
            \begin{itemize}
                \item What exaclty is being optimized? Space or time? 
                \item How often will these operations occur?
                \item Is the graph Directed or undirected? 
                \item Is the graph cyclic or acyclic?
                \item Do edges contain data such as weight or labels? If so, what?
                \item Do vertices contain data such as weight or labels? If so, what?
                \item Is the graph simple or a hyper graph? 
                \item Is the graph dense or sparse? 
                \item Is the graph unipartite, bipartite, or multipartite? 
                \item Do nodes have self-loops?
                \item Is the graph static or dynamic? 
                \item Is the graph complete, just connected, or disconnected?
                \item Are general statistics about the graph known, such as mean and standard
                    deviation of degrees of the nodes and anything else? If so, what are they? 
            \end{itemize}
        }
    \end{homeworkSection}

    \begin{homeworkSection}{2.}
        Provide three scenarios depending on the answers to those questions.\\
        \problemAnswer{
            \begin{enumerate}
                \item Let's start simple; in the first scenario, the graph is simple, undirected,
                    connected, acyclic graph with no loops (in other words a tree), and static. 
                    The data stored in the nodes is quite large so only 1 node can fit in memory at 
                    a time plus roughly 200 KiB for any additional space needed to help index this 
                    and any other nodes. Here, due to the immensity of the data, we optimize for 
                    time when the operations occur. A startup penalty to time is acceptable for 
                    caching. Any node will have a degree anywhere between $0$ and $2$. This tree is 
                    also balanced. Nodes are stored on disk in a single binary file.\\
                \item In the second scenario, the graph is also simple, but is directed, cyclic,
                    sparse, multipartite, contains no self-loops, is quite dynamic, disconnected, 
                    but data points can all be stored in memory easily with plenty of space left 
                    where additional data can be stored for data points. Here we want speed but
                    space has no price. The average degree for the nodes is roughtly $1.2$ with 
                    low variance. Any pair of vertices has at most 1 edge. \\
                \item In the final scenario, the graph is dense and complete (therefore also 
                    cyclic and multi partite), simple, contains self-loops, is directed, and can 
                    be dynamic though this is highly regular and predictable. Also at most one edge
                    exists between any two nodes in the graph. We want both speed and space 
                    efficiency. The completeness also implies the degree of any node is $N-1$ with
                    no variance. \\
            \end{enumerate}
        }
    \end{homeworkSection}

    \begin{homeworkSection}{3.}
        For each scenario, propose the most efficient data structure and supply the time 
        complexity of the requested operations in terms of bigO. What is the space requirement
        for the selected data structure?
        \problemAnswer{
                    \iffalse
            \begin{enumerate}
                \item For the first scenario, since we are dealing with a binary tree, an 
                    efficient implementation would be an array representation where any node,
                    \texttt{n} in the graph located at \texttt{array[n]} would have children (if any 
                    existed for \texttt{n} at \texttt{array[2n]} for the left child and 
                    \texttt{array[2n + 1]} for the right child. Since space is so limited, and 
                    since only one node can fit in memory at a time, we will on startup read
                    through the binary file to read in the offsets of the nodes and node unique keys
                    and cache their addresses (or offsets from the start of the file) and their
                    keys so that \texttt{array[n] = (key,addr)} where \texttt{addr} is the address, 
                    or offset from the start of the file in bytes, of node \texttt{n} while 
                    \texttt{key} is a unique key for the nodes we query with. This array will be
                    used to index the tree to find nodes efficiently without going to disk.\\
                    \textbf{Data Structure} Binary tree indexing the actual graph/tree implemented
                                            as an array.\\
                    \textbf{Time complexity} Adding an edge in this scenario implies also adding a
                                             vertex due to the acyclic nature and limitations on 
                                             degree of nodes. As such they are the same of 
                                             $O(\log n)$. If, over time, nodes are added, this 
                                             complexity could vary from $O(\log n)$ to $O(n)$ should
                                             insertions to the tree cause it to become unbalanced.
                                             Deletion of an edge also implies the deletion of a 
                                             vertex and is also $O(\log n)$ on average and could 
                                             degrade into $O(n)$ if the graph ends up changing
                                             and becoming unbalanced. \\
                    \textbf{Space requirements} Without knowing how large the binary file is, the 
                                             address will need to be stored as an 
                                             \texttt{unsigned long long} for same measure.
                                             Likewise, the unique keys will also need to be
                                             stored as such for similar reasons. On a typical
                                             \texttt{x86\_64} system, \texttt{unsigned long long}
                                             is typically stored as $8$ Bytes. With a graph of 
                                             $10,000$ nodes, the index would then consume 
                                             $10,000 \times 8 \times 2$ Bytes or roughly $156.25$
                                             KiB. \\

                \item For the second scenario, since the graph is sparse and the focus is on 
                    optimizing speed, an optimal implementation is with an adjacency list created
                    via doubly linked circular lists. The doubly linked circular characteristics
                    is to allow the cursor to stay on the last quiried node. This will attempt to 
                    take advantage of temporal and spatial locality.\\
                    \textbf{Data Structure}
                        An adjacency list where each list and sublist are doubly linked circular 
                        lists. Since space is of no concern, nodes will contain the typical doubly
                        circular linked list properties, but will also contain a count for degree.
                        The "main" list is a list of the nodes with pointers to the contained data.
                        The sublists are a listing of nodes with pointers to the nodes in the "main"
                        list wherein a node in the sublist indicates an edge between the two 
                        vertices represented by the main node and the pointed to node. The main node
                        would be the "from" direction of the edge while the pointed-to node is the 
                        "to" direction of the edge. \\
                    \textbf{Time Complexity
                         Add node: 
                             Since order does not matter explicitly within vertices or edges in the 
                             implementation, nodes can be added at any location within the main list
                             of nodes. As such, only adding the node to the list should be $O(1)$ or
                             constant time. \\
                         Remove node:
                             Within the implementation, we can assume a node can only be removed
                             only when the degree of that node is 0. A node can be manually removed
                             but this would require first removing all vertices, the analysis of
                             which is later. As such, node removal is limited to removing the node
                             from the main node list. The worst case scenario is the index to this
                             list points to a node opposite of where the desired node would be 
                             meaning worst case scenario of $O(n)$ with a best case scenario of 
                             $O(1)$ when the index is already pointing to the desired node. \\
                         Add vertex: 
                             Adding a vertex means first traversing to the nodes in questions, but,
                             similar to adding a node, since order of the node and vertex siblings
                             do not matter, the vertex can be added into any location in the vertex
                             list for the node. So traversing the node list in worst case scenario 
                             $O(\frac{n}{2} + \frac{n}{2})$ or $O(n)$ whereas, in the event the
                             nodes are right next to each other, the traversal is then $O(1 + 1)$ or
                             $O(1)$. Adding the vertex into the lists however in all scenarios is 
                             $O(1 + 1)$ or $O(1)$. The total complexity is 
                             $O(\frac{n}{2} + \frac{n}{2} + 1 + 1)$ or $O(n)$. \\
                         Remove vertex:
                             Since the graph is simple, no vertex will have more than $2$ nodes 
                             associated with it. As such, similar to adding a vertex, the node 
                             list must be travesed for each meaning this operation is $O(n)$ in
                             the worst case scenario and $O(1)$ in best case scenario. However, 
                             removing a vertex means traversing both lists of vertecies for the 
                             nodes. The worst case scenario then becomes 
                             $O(max_degree(G) + max_degree(G))$. The best case however is when the index is already
                             pointing to the vertecies, so, similar to removing nodes, this becomes
                             $O(1)$.  \\
                    \textbf{Space Requirements}
                        The adjacency list would need a "main" node list each with a list of
                        vertices. Below are basic mockups for the structure of the nodes;
                        \begin{lstlisting}[language=C++]
                            struct vertex_node {
                                struct *vertex_node prev;
                                struct *vertex_node next;
                                void *data;
                                unsigned int degree;
                            }

                            struct edge_node {
                                struct *edge_node prev;
                                struct *edge_node next;
                                struct *vertex_node;
                            }
                        \end{lstlisting}
                     \\
            \end{enumerate}
                     \fi
            
        }
    \end{homeworkSection}

\end{homeworkProblem}

%----------------------------------------------------------------------------------------
%	PROBLEM 2
%----------------------------------------------------------------------------------------

\begin{homeworkProblem}[Problem \#\arabic{homeworkProblemCounter}]
Let $G$ be a simple directed graph with $N$ nodes. The path matrix of $G$ is the $n$-­square matrix
$P$ defined as follows: Each $(i,j)$ element of the matrix is $1$ if there is a path from node $i$
to node $j$ in $G$.

%--------------------------------------------

    \begin{homeworkSection}{1.} % Section within problem
        Using only an adjacency matrix $A$ of graph $G$ defined as a boolean $(0,1)$ matrix, design
        an algorithm that constructs the matrix $P$ in the most efficient manner. Write any
        mathematical relationship for $P$ you will use for your algorithm and justify this relationship
        (formal or informal proof is up to you). [Hint: What is $A$ times $A$? What is $A$ times $A$ times
        $A$ ?]\\
        \vspace{10pt} % Question
        \problemAnswer{
            MEH I HATE X11
        }
    \end{homeworkSection}

    \begin{homeworkSection}{2.} % Section within problem
        Provide a pseudo­code of the algorithm.
        \vspace{10pt} % Question
        \problemAnswer{
            MEH I HATE X11
        }
    \end{homeworkSection}

    \begin{homeworkSection}{3.} % Section within problem
        Analyse the time complexity of your algorithm in terms of big­O.
        \vspace{10pt} % Question
        \problemAnswer{
            MEH I HATE X11
        }
    \end{homeworkSection}

    \begin{homeworkSection}{4.} % Section within problem
        Define the values of N, for which this algorithm becomes impractical (e.g., will take more
        than a day to get an answer) on my laptop running at $2.5$ GHz (i.e., $2.4 * 10^9$ floating
        point operations per second).
        \vspace{10pt} % Question
        \problemAnswer{
            MEH I HATE X11
        }
    \end{homeworkSection}

    \begin{homeworkSection}{4.} % Section within problem
        If my laptop has 512 MB, 1GB, or 16 GB of RAM, will your answer to (4) change given
        my memory constraints (i.e., in terms of big­O for space requirements) and how many
        bytes do you require per your selected data structure; crude estimation with the
        justification is enough?
        \vspace{10pt} % Question
        \problemAnswer{
            MEH I HATE X11
        }
    \end{homeworkSection}
\end{homeworkProblem}

%--------------------------------------------


%----------------------------------------------------------------------------------------
%	PROBLEM 3
%----------------------------------------------------------------------------------------

\begin{homeworkProblem}[Prob. \Roman{homeworkProblemCounter}] % Roman numerals

%--------------------------------------------

\begin{homeworkSection}{\homeworkProblemName:~(a)} % Using the problem name elsewhere
\problemAnswer{ % Answer
}
\end{homeworkSection}

%--------------------------------------------

\begin{homeworkSection}{\homeworkProblemName:~(b)}
\vspace{10pt} % Question

\problemAnswer{ % Answer
}
\end{homeworkSection}

%--------------------------------------------

\end{homeworkProblem}

%----------------------------------------------------------------------------------------
%	PROBLEM 4
%----------------------------------------------------------------------------------------

\begin{homeworkProblem}[Prob. \Roman{homeworkProblemCounter}] % Roman numerals
\problemAnswer{ % Answer
}
\end{homeworkProblem}

%----------------------------------------------------------------------------------------

\end{document}
