%%%%%%%%%%%%%%%%%%%%%%%%%%%%%%%%%%%%%%%%%
% Structured General Purpose Assignment
% LaTeX Template
%
% This template has been downloaded from:
% http://www.latextemplates.com
%
% Original author:
% Ted Pavlic (http://www.tedpavlic.com)
%
% Note:
% The \lipsum[#] commands throughout this template generate dummy text
% to fill the template out. These commands should all be removed when 
% writing assignment content.
%
%%%%%%%%%%%%%%%%%%%%%%%%%%%%%%%%%%%%%%%%%

%----------------------------------------------------------------------------------------
%	PACKAGES AND OTHER DOCUMENT CONFIGURATIONS
%----------------------------------------------------------------------------------------

\documentclass{article}

\usepackage{fancyhdr} % Required for custom headers
\usepackage{lastpage} % Required to determine the last page for the footer
\usepackage{extramarks} % Required for headers and footers
\usepackage{graphicx} % Required to insert images
\usepackage{lipsum} % Used for inserting dummy 'Lorem ipsum' text into the template

% Margins
\topmargin=-0.45in
\evensidemargin=0in
\oddsidemargin=0in
\textwidth=6.5in
\textheight=9.0in
\headsep=0.25in 

\linespread{1.1} % Line spacing

% Set up the header and footer
\pagestyle{fancy}
\lhead{\hmwkAuthorName} % Top left header
\chead{\hmwkClass\ (\hmwkClassInstructor\ \hmwkClassTime): \hmwkTitle} % Top center header
\rhead{\firstxmark} % Top right header
\lfoot{\lastxmark} % Bottom left footer
\cfoot{} % Bottom center footer
\rfoot{Page\ \thepage\ of\ \pageref{LastPage}} % Bottom right footer
\renewcommand\headrulewidth{0.4pt} % Size of the header rule
\renewcommand\footrulewidth{0.4pt} % Size of the footer rule

\setlength\parindent{0pt} % Removes all indentation from paragraphs

%----------------------------------------------------------------------------------------
%	DOCUMENT STRUCTURE COMMANDS
%	Skip this unless you know what you're doing
%----------------------------------------------------------------------------------------

% Header and footer for when a page split occurs within a problem environment
\newcommand{\enterProblemHeader}[1]{
\nobreak\extramarks{#1}{#1 continued on next page\ldots}\nobreak
\nobreak\extramarks{#1 (continued)}{#1 continued on next page\ldots}\nobreak
}

% Header and footer for when a page split occurs between problem environments
\newcommand{\exitProblemHeader}[1]{
\nobreak\extramarks{#1 (continued)}{#1 continued on next page\ldots}\nobreak
\nobreak\extramarks{#1}{}\nobreak
}

\setcounter{secnumdepth}{0} % Removes default section numbers
\newcounter{homeworkProblemCounter} % Creates a counter to keep track of the number of problems

\newcommand{\homeworkProblemName}{}
\newenvironment{homeworkProblem}[1][Problem \arabic{homeworkProblemCounter}]{ % Makes a new environment called homeworkProblem which takes 1 argument (custom name) but the default is "Problem #"
\stepcounter{homeworkProblemCounter} % Increase counter for number of problems
\renewcommand{\homeworkProblemName}{#1} % Assign \homeworkProblemName the name of the problem
\section{\homeworkProblemName} % Make a section in the document with the custom problem count
\enterProblemHeader{\homeworkProblemName} % Header and footer within the environment
}{
\exitProblemHeader{\homeworkProblemName} % Header and footer after the environment
}

\newcommand{\problemAnswer}[1]{ % Defines the problem answer command with the content as the only argument
\noindent\framebox[\columnwidth][c]{\begin{minipage}{0.98\columnwidth}#1\end{minipage}} % Makes the box around the problem answer and puts the content inside
}

\newcommand{\homeworkSectionName}{}
\newenvironment{homeworkSection}[1]{ % New environment for sections within homework problems, takes 1 argument - the name of the section
\renewcommand{\homeworkSectionName}{#1} % Assign \homeworkSectionName to the name of the section from the environment argument
\subsection{\homeworkSectionName} % Make a subsection with the custom name of the subsection
\enterProblemHeader{\homeworkProblemName\ [\homeworkSectionName]} % Header and footer within the environment
}{
\enterProblemHeader{\homeworkProblemName} % Header and footer after the environment
}
   
%----------------------------------------------------------------------------------------
%	NAME AND CLASS SECTION
%----------------------------------------------------------------------------------------

\newcommand{\hmwkTitle}{Assignment\ \#1} % Assignment title
\newcommand{\hmwkDueDate}{Monday,\ January\ 1,\ 2012} % Due date
\newcommand{\hmwkClass}{BIO\ 101} % Course/class
\newcommand{\hmwkClassTime}{10:30am} % Class/lecture time
\newcommand{\hmwkClassInstructor}{Jones} % Teacher/lecturer
\newcommand{\hmwkAuthorName}{John Smith} % Your name

%----------------------------------------------------------------------------------------
%	TITLE PAGE
%----------------------------------------------------------------------------------------

\title{
\vspace{2in}
\textmd{\textbf{\hmwkClass:\ \hmwkTitle}}\\
\normalsize\vspace{0.1in}\small{Due\ on\ \hmwkDueDate}\\
\vspace{0.1in}\large{\textit{\hmwkClassInstructor\ \hmwkClassTime}}
\vspace{3in}
}

\author{\textbf{\hmwkAuthorName}}
\date{} % Insert date here if you want it to appear below your name

%----------------------------------------------------------------------------------------

\begin{document}

\maketitle

%----------------------------------------------------------------------------------------
%	TABLE OF CONTENTS
%----------------------------------------------------------------------------------------

%\setcounter{tocdepth}{1} % Uncomment this line if you don't want subsections listed in the ToC

\newpage
\tableofcontents
\newpage

%----------------------------------------------------------------------------------------
%	PROBLEM 1
%----------------------------------------------------------------------------------------

% To have just one problem per page, simply put a \clearpage after each problem

\begin{homeworkProblem}[Problem 1]
    Suppose you built the MIN­HEAP for a graph with the node weights corresponding to the node
    degree in the graph. The root of the binary heap has the node with the MINIMUM degree in the
    graph, hence, MIN­HEAP. (Note: node and vertex are used interchangeably in this course.)
\begin{homeworkSection}{1}
    What is the cost of finding the node with the smallest degree using this ADT?
    \vspace{10pt}

    \problemAnswer {
        A lot. 
    }
\end{homeworkSection}

\begin{homeworkSection}{2}
    What is the cost of finding the node with the largest degree using this ADT? Justify your
    answer.
    \vspace{10pt} % Question
    \problemAnswer {
        Even more
    }
\end{homeworkSection}

\begin{homeworkSection}{3}
    Suppose a high authority figure (a customer) from the U.S. CDC (Center for Disease
    Control) requests you to write a simulation program for estimating the spread (the
    number of infected people) of the virus propagation before CDC runs out of the vaccines.
    CDC will provide you the information about the social network (SN) of people who could
    be in direct contact. According to CDC, the number K of available vaccines is limitted,
    namely K << N, where N is the number of people in SN. Your immunization strategy
    for selecting which people to immunize is based on the degree of the node in the
    SN. At any time step of the simulation, only one person can be immunized. Once the
    person is rendered the vaccine, he/she can not infect the people he/she could be in
    touch with according to SN information.\\

    Note: there is no correct/incorrect answer to this problem. The purpose is to give you an
    opportunity to experience the complexity of solving a realistic problem and to assess the
    reasoning you provide to justify your approach to this problem.

    \begin{homeworkSection}{(a)}
        Given your current MIN­HEAP ADT implementation (C code is given in lecture
        notes on binary heaps), write the most efficient pseudo­code for this simulation.
        \vspace{10pt}

        \problemAnswer {
            DONT
        }
    \end{homeworkSection}

    \begin{homeworkSection}{(b)}
        Clearly articulate all the assumptions of your model.
        \vspace{10pt}

        \problemAnswer {
            It makes an ass out of you and me
        }
    \end{homeworkSection}

    \begin{homeworkSection}{(c)}
        Provide big­O analysis of time complexity of your simulation.
        \vspace{10pt}

        \problemAnswer {
            It makes an ass out of you and me
        }
    \end{homeworkSection}

    \begin{homeworkSection}{(d)}
        Discuss pros and cons of your simulation program.
        \vspace{10pt}

        \problemAnswer {
            Pro professional
        }
    \end{homeworkSection}
\end{homeworkSection}
\end{homeworkProblem}

%----------------------------------------------------------------------------------------
%	PROBLEM 2
%----------------------------------------------------------------------------------------

\begin{homeworkProblem}[Exercise \#\arabic{homeworkProblemCounter}] % Custom section title
\lipsum[3] % Question

%--------------------------------------------

\begin{homeworkSection}{(a)} % Section within problem
\lipsum[4]\vspace{10pt} % Question

\problemAnswer{ % Answer
\lipsum[5]
}
\end{homeworkSection}

%--------------------------------------------

\begin{homeworkSection}{(b)} % Section within problem
\problemAnswer{ % Answer
\lipsum[6]
}
\end{homeworkSection}

%--------------------------------------------

\end{homeworkProblem}

%----------------------------------------------------------------------------------------
%	PROBLEM 3
%----------------------------------------------------------------------------------------

\begin{homeworkProblem}[Prob. \Roman{homeworkProblemCounter}] % Roman numerals

%--------------------------------------------

\begin{homeworkSection}{\homeworkProblemName:~(a)} % Using the problem name elsewhere
\problemAnswer{ % Answer
\lipsum[7]
}
\end{homeworkSection}

%--------------------------------------------

\begin{homeworkSection}{\homeworkProblemName:~(b)}
\lipsum[8]\vspace{10pt} % Question

\problemAnswer{ % Answer
\lipsum[9]
}
\end{homeworkSection}

%--------------------------------------------

\end{homeworkProblem}

%----------------------------------------------------------------------------------------
%	PROBLEM 4
%----------------------------------------------------------------------------------------

\begin{homeworkProblem}[Prob. \Roman{homeworkProblemCounter}] % Roman numerals
\problemAnswer{ % Answer
\lipsum[10]
}
\end{homeworkProblem}

%----------------------------------------------------------------------------------------

\end{document}
